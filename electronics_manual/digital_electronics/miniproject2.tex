\section*{Mini-Project 2: A Two-Bit Adder}
\label{lab_proportional_controller}

\fancypagestyle{interlude}[fancy]{%
	\fancyhead[LO,RE]{}
	\fancyhead[LE,RO]{\slshape \MakeUppercase{Mini-Project 2: A Two-Bit Adder}}
}
\addcontentsline{toc}{section}{\hspace{0.35in} \textbullet~Mini-Project 2: A Two-Bit Adder}
\pagestyle{interlude}


%\makelabheader %(Space for student name, etc., defined in master.tex)

\bigskip

In this Mini-project, you will combine several ideas from previous labs to make a circuit that is significantly more complex than anything you've built so far.  It will take some time, and you can continue moving forward with the next lab as you work on this in the background.  

\medskip

Logic circuits can be used to do arithmetic on binary (base two) numbers.  Design and build a circuit that can add two 2-bit numbers.  Each of the inputs can be either 0, 1, 2, or 3, represented by two input bits as 00, 01, 10, or 11.  The output will be the binary representation from 0 to 6, using three output bits.  Test your circuit, using the logic switches on your board to control the inputs, and the logic indicators to view the outputs.

