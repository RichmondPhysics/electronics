\section{List of Circuits to Know}
\label{circuits_to_know}

On an exam, you will be expected to know how to design each of the following circuits.  You will need to draw a circuit diagram, \textit{from memory}, and calculate the values of all components (resistors, capacitors, etc.) needed to achieve specified characteristics like a required gain, cutoff frequency, output voltage, or whatever.

\begin{flushleft}
\begin{enumerate}[align=left,leftmargin=*]%[wide]

\item (Lab~\ref{lab_op-amps}, part~\ref{part_non-inverting_amplifier}): A non-inverting amplifier with a specificed gain (say, a gain of 10) using an op-amp such as the LM324.

\item (Lab~\ref{lab_op-amps}, part~\ref{part_inverting_amplifier}): An inverting amplifier with a specificed gain.

\item (Lab~\ref{lab_op-amps}, part~\ref{part_summer}): A summing amplifier with a specified set of weighted gains for each input: $V_{\rm out}=G_1V_1 + G_2V_2 + ...$ .

\item (Lab~\ref{lab_op-amps}, part~\ref{part_differential}): A differential amplifier with a specificed gain, using an op-amp like the LM324.

\item (Lab~\ref{lab_digital_electronics}, part~\ref{part_schmitt_trigger}): A Schmitt trigger that switches from low to high and high to low at specified input voltages.

\item (Lab~\ref{lab_capacitors}, part~\ref{part_delay_switch}): A delay switch, using an LM311 comparator and a capacitor, which switches the output from low to high or high to low a specified delay time after the input is changed.

\item (Lab~\ref{lab_capacitors}, part~\ref{part_square_generator}): A square wave generator, which generates an output of a specified frequency, using an LM311 comparator and a capacitor.

\end{enumerate}
\end{flushleft}