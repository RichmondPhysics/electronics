\section{List of Circuits to Know}
\label{circuits_to_know}

On an exam, you will be expected to know how to design each of the following circuits.  You will need to draw a circuit diagram, \textit{from memory}, and calculate the values of all components (resistors, capacitors, etc.) needed to achieve specified characteristics like a required gain, cutoff frequency, output voltage, or whatever.

\begin{flushleft}
\begin{enumerate}[align=left,leftmargin=*]%[wide]

\item (Lab~\ref{lab_op-amps}, part~\ref{part_non-inverting_amplifier}): A non-inverting amplifier with a specificed gain (say, a gain of 10) using an op-amp such as the LM324.

\item (Lab~\ref{lab_op-amps}, part~\ref{part_inverting_amplifier}): An inverting amplifier with a specificed gain.

\item (Lab~\ref{lab_op-amps}, part~\ref{part_summer}): A summing amplifier with a specified set of weighted gains for each input: $V_{\rm out}=G_1V_1 + G_2V_2 + ...$.

\item (Lab~\ref{lab_op-amps}, part~\ref{part_differential}): A differential amplifier with a specificed gain, using an op-amp like the LM324.

\item (Lab~\ref{lab_digital_electronics}, part~\ref{part_schmitt_trigger}): A Schmitt trigger that switches from low to high and high to low at specified input voltages.

\item (Lab~\ref{lab_capacitors}, part~\ref{part_delay_switch}): A delay switch, using an LM311 comparator and a capacitor, which switches the output from low to high or high to low after a specified delay time from when the input is changed.

\item (Lab~\ref{lab_capacitors}, part~\ref{part_square_generator}): A square wave generator, using an LM311 comparator and a capacitor, which generates an output of a specified frequency.

\item (Lab~\ref{lab_diodes}, part~\ref{part_series_clipper}): A series clipper, also known as a half-wave rectifier.

\item (Lab~\ref{lab_diodes}, part~\ref{part_shunt_clipper}): A shunt clipper, in which the current or power dissipated in the diode is limited to a specified amount.

\item (Lab~\ref{lab_diodes}, part~\ref{part_biased_shunt_clipper}): A biased shunt clipper with a specified $V_{\rm max}$ or $V_{\rm min}$.

\item (Lab~\ref{lab_diodes}, parts~\ref{part_zener_shunt_clipper} and \ref{part_double_zener_shunt_clipper}): A Zener shunt clipper or double Zener shunt clipper with a specified $V_{\rm max}$ or $V_{\rm min}$ and a specified $I_{\rm max}$ through the diodes.

\item (Lab~\ref{lab_power_supply}, part~\ref{part_full_wave_bridge_rectifier}): A full-wave bridge rectifier.

\item (Lab~\ref{lab_power_supply}, part~\ref{part_linear_power_supply_78xx}): A linear power supply using a full-wave bridge rectifier, a capacitor, and a 78xx series voltage regulator to produce a DC output voltage with a specified maximum ripple voltage at a given load current.

\item (Lab~\ref{lab_filters}, parts~\ref{part_low_pass_RC} and \ref{part_low_pass_RL}): A low pass RC or RL filter with a specified cutoff frequency.

\item (Lab~\ref{lab_filters}, parts~\ref{part_high_pass_RC} and \ref{part_high_pass_RL}): A high  pass RC or RL filter with a specified cutoff frequency.

\item (Lab~\ref{lab_filters}, part~\ref{part_band_pass_filter}): A band pass filter with specified high and low cutoff frequencies.

\item (Lab~\ref{lab_filters}, part~\ref{part_DC_bias}): A circuit using a capacitor and two resistors to add a specified DC bias to an AC signal above a given cutoff frequency.

\item (Lab~\ref{lab_filters}, part~\ref{part_resonance_filter}): An LC resonance  filter with specified resonance frequency.

\item (Lab~\ref{lab_bjt}, part~\ref{part_buffer_op-amp_emitter_follower}): A buffer amplifier using an op-amp and a bipolar junction transistor as an emitter follower.

\item (Lab~\ref{lab_bjt}, part~\ref{part_biased_output_and_input_correct}): An emitter follower amplifier for AC, for a specified frequency range.

\item (Lab~\ref{lab_multisim}, part~\ref{part_common_emitter_and_emitter_follower}): A multistage AC amplifier consisting of a common emitter stage with a specified gain and an emitter follower stage for high current output.

\item (Lab~\ref{lab_microphone_photocell}, part~\ref{part_microphone}): A circuit to obtain an output from an electret microphone with an internal JFET. 

\item (Lab~\ref{lab_timers}, part~\ref{part_one_shot_timer}): A one-shot timer using an NE555 timer to hold an output high for a specified time.

\item (Lab~\ref{lab_timers}, part~\ref{part_astable}): A square wave generator with a specified output frequency using an NE555 timer.

\item (Lab~\ref{lab_timers}, part~\ref{part_square_wave_duty_cycle}): A square wave generator with a specified duty cycle using an NE555 timer and an extra diode. 

\item (Lab~\ref{lab_counters}, parts~\ref{part_two_digit_counter_99} and \ref{part_two_digit_counter_39}): A two-digit counter, using a CD74HC390E dual counter, a 74LS47 decoder, and two MAN72 seven-segment displays to count from 0 to any arbitrary number.
\end{enumerate}
\end{flushleft}