\section{The Oscilloscope is Our Friend!}
\label{lab_oscilloscope}

%\makelabheader %(Space for student name, etc., defined in master.tex)

\bigskip

\begin{enumerate}[wide]

\item Turn on your oscilloscope, and hook a probe into the BNC connector of the scope marked ``CH 1'' (Channel 1).  Use it to measure the output of a 1 kHz sine wave from your function generator, with the amplitude cranked up high (a maximum amplitude of about 10 or 12 volts).  What does the VOLTS/DIV knob do?  What does the ``Div'' mean?

\item Your probe has a little alligator clip at the end of it, which you probably connected to ground.  Try disconnecting it.  Why does it not matter here?

\item There's a tiny little switch on your probes, marked ``1X'' and ``10X.'' (I'm \emph{not} talking about the menu item on the scope screen, but the physical switch at the end of the actual probes.)  What does this switch do?

\item The menu on your scope's screen should show various options, including ``coupling,'' ``BW Limit,'' ``Volts/Div,'' and ``Probe.''  (If it doesn't show that, probably because you were already playing with other buttons, hit the ``Ch1 Menu'' button.)  What does hitting the button next to ``Probe'' do?  How is this related to the switch on the side of your probes?

\item What does the SEC/DIV knob do?  Measure the period of the sine wave by counting divisions on the screen.  Is it consistent with a 1 kHz sine wave?

\item Sketch the shape of the waveform from your function generators for a triangle wave and a square wave.  For a square wave, use your scope to measure about how long it takes for the voltage to rise from its negative value to its peak positive value.  (Is it really instantaneous?)

\item Go back to a sine wave, and hit the ``measure'' button.  Hit the first menu button (to the right of the screen), and then under ``Type'' have it measure ``Mean.''  Hit the ``Back'' option, and set up the second menu button to measure peak-to-peak voltage.  Set up the third measurement for ``RMS,'' a fourth one for ``MAX,'' and a fifth one for ``MIN.'' Show in a sketch which parts of the sine wave these are measuring.  Which of these correspond to the measurements your DMM makes when it measures AC voltage?

\item How do the values of those measurements change for triangle and square waves?  What the heck is RMS, anyway?

\item What do the knobs marked ``position'' do?  (Vertical and Horizontal.)  Notice on screen how you can tell when they're back to zero. 

\item Use your second probe to measure the function generator output marked ``TTL.'' Plug it into channel 2, and hit the blue ``Ch 2 menu'' button to turn on the second display.  How is the TTL output affected when you adjust the amplitude and frequency of the sine wave (or triangle or square wave)?

\item Set your function generator back to a nice big 1 kHz sine wave, and hit the ``Trig menu'' button.  (``Trig'' stands for ``Trigger.'')  Cycle through all of the options under ``Source'' and describe what you see.  What the heck does ``triggering'' mean?  Back on ``Ch1'' for source, play with the ``level'' knob, and describe what happens.

\item With the level adjustment back on zero, adjust the amplitude of your function generator down to a point where the triggering really flips out and can't give you a decent looking clean sine wave.   At this point, when you would normally despair of ever being able to get a clean measurement, go to the trigger menu, and select ``Ch2'' under source.  You may also need to adjust the trigger level slightly.  Why is it better to trigger off of the TTL output than a sine wave?

\item Under trigger sources, after ``Ch1'' and ``Ch2,'' what does ``Ext'' mean?  (Hint: unplug the probe from channel 2, and plug it into the jack marked ``Ext Trig.'')  

\item Plug the 2nd probe (measuring the TTL output) back into channel two, and go back to using channel 1 as a trigger.  Crank up the amplitude to get a clean signal.  Now hit the ``Ch~2'' menu, and start playing with the on-screen options.   What does ``Invert'' do?  What does ``Coupling'' do? 

\item While you have pretty traces of Ch1 and Ch2 on your screens, hit the Cursor button, select ``Voltage'' under Type, and be sure the ``Source'' is set to either Ch1 or Ch2, whatever you have on your screen.  Try twiddling the vertical position knobs.  Cool, eh?  What do the various other options do?

\item Use AC coupling to take a careful look at the output of your 5 Volt DC power supply.  If you crank up the sensitivity of the scope using the volts/div knob, you should be able to see in great detail the small AC part that lies on top of the 5 volt DC signal.  Adjust the time scale to get a good picture of what the AC part is really doing.  What is the characteristic frequency (or frequencies) of this AC part?  Where might they be coming from?

\bigskip

\textit{If you ever find yourself unable to get what you want from your oscilloscope, Appendix \ref{oscilloscope_checklist} provides a handy checklist of things to consider.}

\end{enumerate}


\textbf{Possible Exam Questions:}

\begin{itemize}

\item A signal varies sinusoidally between +4 volts and +6 volts.  Describe what the oscilloscope trace would look like for AC coupling, DC coupling, and ground.

\item Describe how to use the TTL output of your function generator to get good triggering for a precise measurement of a small signal. What inputs would you use, and how would you trigger? 

\end{itemize}





